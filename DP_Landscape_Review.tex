% F1000Research template from writeLaTeX

%\documentclass[10pt,a4paper,twocolumn]{article}
\documentclass[10pt,twocolumn]{article}

\usepackage[superscript]{cite}
\usepackage{f1000_styles}
\usepackage{hyperref}

\hypersetup{colorlinks=true}


\begin{document}


\title{The Data Publication Landscape.}
\author[1]{John Ernest Kratz}
\author[1]{Carly Strasser}
\affil[1]{California Digital Library, University of California Office of the President, Oakland, CA 94612, USA}

\maketitle
\thispagestyle{fancy}

% Please list all authors that played a significant role in the research involved in the article. Please provide full affiliation information (including full institutional address, ZIP code and e-mail address) for all authors, and identify who is/are the corresponding author(s).

\begin{abstract}


The movement to weave datasets into the scholarly record as first class research products (validated, preserved, cited, and credited) has been inching forward for some time, but now the pace is quickening.
Even as data publication venues proliferate, there remains significant debate over formats, processes, and terminology.
Here, we present an overview of data publication initiatives underway and the current conversation, highlighting points of consensus and issues still in contention.
Data publication implementations differ in, among other factors, the kind of documentation, the location of the documentation relative to the data, and how the data is validated.
Publishers may present the data as supplemental material to a journal article, with a descriptive ``data paper'', or independently.
Further complicating the situation, different initiatives and communities use the same terms to refer distinct but overlapping concepts.
For instance, the term ``published'' means that the data is publicly available and citable to virtually everyone, but it may or may not mean that the data has been peer-reviewed.
In turn, data ``peer-review'' standards and processes encompass the full range employed in reviewing the literature, plus some novel variations.
Basic data citation is a point of consensus, but the general agreement on the core elements of a dataset citation frays if the data is dynamic or part of a larger set.
Even before the term data publication solidifies, some are looking past publication to other metaphors, notably `data as software', for answers to the still-open questions.

\end{abstract}
\clearpage



\section*{Introduction: what does data publication mean?}\label{introduction}

The idea that researchers should share data to advance knowledge and promote the common good is an old one, but in recent years the conversation has shifted from sharing data to ``publishing'' data.\cite{costello_motivating_2009,smith_data_2009,lawrence_data_2011}
The shift in language is driven by the conviction that datasets should be brought into the scholarly record and afforded the same first class status as traditional research products like journal articles.\cite{sarah_callaghan_making_2012}
While this goal is widely shared within the scholarly communication community, different people and organizations imply different things by the phrase \emph{data publication}. %meg: Is it?

The community largely agrees on two essential properties of a data publication\cite{smith_data_2009,sarah_callaghan_making_2012}.
First, published data is publicly \textbf{available} now and for the indefinite future; access might require payment of fees or acceptance of a legal agreement, but not the approval of the author.
Second, like a book or journal article, a data publication can be formally \textbf{cited}.
Open questions surround a third property: how and to what extent a published dataset must be \textbf{validated}.
Callaghan (2012)\cite{sarah_callaghan_making_2012} draws a distinction between data that has been shared, published (lower-case “p”), or Published (upper-case “P”): \textbf{shared} data is available, \textbf{published} data is available and citable, and \textbf{Published} data is available, citable, and validated.
In practice, availability is usually satisfied by depositing the dataset in a repository, citability by assigning a persistent identifier (e.g. a Digital Object Identifier, or DOI), and validity by peer-review.

\subsection*{Why publish data?}\label{why-publish-data}

The underlying goals of data publication are to enable research \textbf{reproduction} and to facilitate data \textbf{reuse}.
Hidden primary data exacerbates science's very public ``reproducibility crisis''\cite{mobley_survey_2013,pashler_is_2012,zimmer_rise_2012,hiltzik_science_2013,begley_drug_2012}.
Widespread publication of the data underlying research papers could help expose both fraud and honest errors\cite{drew_lost_2013}.
The leaders of the US National Institutes of Health (NIH) recently cited ``provid[ing] greater transparency of the data that are the basis of published manuscripts'' as one way to improve scientific reproducibility\cite{collins_nih_2014}.
%meg: give a colorful example

Journals frequently require authors to supply underlying data on request already.
As of 2011, 88\% of high-impact publications required a statement regarding the availability of underlying data; half of those made willingness to provide data a condition of publication.\cite{alsheikh-ali_public_2011}
However, the authors of 59\% of papers examined failed to adhere to the availability instructions.
Vines et al. (2014)\cite{vines_availability_2014} could only obtain underlying data from 101 of 516 papers published from 1991 to 2011.
Availability dropped off sharply with time; data could only be obtained from only two of the 62 oldest papers.
Now, some journals are beginning to require that underlying data be published simultaneously with the article.
In 2010, a coalition of Ecology and Evolutionary Biology journals began to require that the data underlying articles be archived with a maximum embargo of one year.\cite{whitlock_data_2010, fairbairn_advent_2010}
\href{http://f1000research.com}{\emph{F1000Research}} has had a similar policy (without an embargo period) since its inception, and the {\href{http://www.plos.org/}{Public Library of Science (PLOS)}} journals followed suit this year.\cite{bloom_data_2014}

Although there can be no substitute for investing time and funding in new data collection, appropriate data reuse lowers costs and accelerates research.
Documenting, publishing, and archiving data is time consuming and costly, but far less so than repeating the data collection.
For example, \href{http://opencontext.org/}{Open Context} published archaeological data from a site in eastern Turkey at the substantial cost of \$10,000-15,0000, but this publication expense was still minor compared to \$800,000 spent to collect the data\cite{kansa_we_2014}.
Piwowar (2011) contrasted the impact of \$100,000 in National Science Foundation (NSF) grants, which generates an average of three to four papers, with an estimate that the same investment in curating, archiving, and publishing data could contribute to over 1,000 publications.\cite{piwowar_data_2011}
Furthermore, while some data is merely expensive to recreate, time-dependent or ephemeral data, (e.g. climate records or observations of unique astronomical events) can never be recreated for any price.\cite{gray_online_2002}

\section*{Types of data publication}\label{types-of-data-publication}

The still-solidifying phrase ``data publication'' covers diverse kinds of research objects published via a variety of processes.
Depending on the speaker, a data publication might be an excel spreadsheet on a website, a set of images in an institutional archive, a stream of readings from a weather station transmitted over the internet, or a peer-reviewed article describing a dataset hosted elsewhere.
Different disciplines, subdisciplines, and individual researchers consider different assortments of digital material to be their data, and it is unlikely that any single structure will be ideal for every discipline and dataset.
But, we can hope to agree on a manageable number of blueprints.
For instance, Lawrence (2011) describes five data publication models ``discriminated in the main by how the roles involved in publication are distributed between the various actors'' (e.g. the author, archive or journal).\cite{lawrence_data_2011}
Here, we will more simply group data publications into three categories based on the accompanying documentation; a dataset may \textbf{supplement} a traditional research paper, be the \textbf{subject} of a ``data paper'', or be \textbf{independent} of any paper.

\subsection*{Data that supplements a paper}\label{paper-supplement-data}

For researchers, the most familiar kind of data publication is a traditional journal article accompanied by the underlying data.
The data can be hosted by the journal publisher as supplementary material or deposited in a third-party repository.
The trend is away from supplemental material because repositories are generally considered to be better suited to ensure long-term preservation and access to the data.
For instance, \emph{The Journal of Neuroscience} stopped publishing supplemental material in 2010; the announcement promotes disciplinary repositories as ``vastly superior to supplemental material as a mechanism for disseminating data''\cite{maunsell_announcement_2010}.
Data underlying any peer-reviewed or otherwise ``reputable'' publication can be deposited in the \href{http://datadryad.org/}{Dryad} repository.
Dryad makes data available and citable, but the publisher of the article manages any assessment of scientific validity.
Other third-party repositories include \href{http://figshare.com/}{figshare}, \href{http://zenodo.org/}{Zenodo}, institutional repositories (e.g. the \href{https://purr.purdue.edu/}{Purdue Research Repository}), and discipline-specific repositories (e.g. DNA sequences are deposited in \href{http://www.ncbi.nlm.nih.gov/genbank/}{GenBank}\cite{benson_genbank_2013} and protein structures in the \href{http://www.rcsb.org/}{Protein Data Bank}\cite{berman_protein_2000}).


\subsection*{Data as the subject of a paper}\label{paper-subject-data}

A \textbf{data paper} describes a dataset by thoroughly detailing the rationale and collection methods, but lacks any analysis or conclusions \cite{newman_data_2009}.
Data papers are flourishing as a new article type in journals such as \emph{F1000Research}, \href{http://www.internetarchaeology.org/}{\emph{Internet Archaeology}}, and \href{http://www.gigasciencejournal.com/}{\emph{GigaScience}}\cite{gigascience}, as well as in new dedicated journals like \href{http://onlinelibrary.wiley.com/journal/10.1002/%28ISSN%292049-6060}{\emph{Geoscience Data Journal}}\cite{geoscience_data_journal}, Nature Publishing Group's \href{http://www.nature.com/scientificdata/}{\emph{Scientific Data}}, and a trio of ``metajournals'' from Ubiquity Press.

Data papers' length and structure varies between journals, but the tendency is toward short, tightly structured formats.
All journals require an abstract, collection methods, and a description of the dataset; a few encourage authors to suggest potential uses for the data (e.g. \emph{Internet Archaeology}, and \href{http://openhealthdata.metajnl.com/about/submissions#authorGuidelines}{\emph{Open Health Data}}).
Some journals supplement this general framework with field-specific sections; for instance, \emph{Internet Archaeology} and the \href{http://openarchaeologydata.metajnl.com/}{\emph{Journal of Open Archaeology Data}} both include a section to describe temporal and geographic scope.
Data papers are most crisply defined not by the presence of any particular information, but by the absence of analysis or conclusions.
A sharp distinction from other article types is important because many journals do not consider a data paper to be prior publication if the authors seek to publish an analysis of the same dataset (e.g. Nature-titled journals, \emph{Science}, and others queried by \href{https://f1000research.com/data-policies}{F1000Research}).

Data journals generally limit themselves to publishing the descriptive paper; a trusted repository publishes the data itself.
For instance, \emph{Scientific Data} and \emph{Geoscience Data Journal} each direct authors to a list of approved repositories.
As an exception, \emph{GigaScience} hosts data in an integrated repository named \emph{GigaDB}.
An early implementer of data papers, \emph{The International Journal of Robotics Research}\cite{international_journal_of_robotics_research}\cite{newman_data_2009} is unusual in permitting authors to host datasets on their own websites.


\subsection*{Data independent of any paper}\label{paper-independent-data}

To be useful or reproducible, a dataset must be accompanied by descriptive information (i.e. metadata)\cite{gray_online_2002}, but this need not take the form of a journal article.
Instead, some repositories publish rich, structured and/or freeform description together with the data, and the distinction between a data repository and a data publisher can be blurry.
Repositories provide access and citability, but the degree of validation varies widely and few are equipped to provide peer-review.
To make data publication as easy as possible for authors, figshare and \href{https://zenodo.org/}{ZENODO} publish datasets from any field with minimal validation.

\section*{Availability}\label{availability}

Fundamentally, to publish is to make public, and to publish data is to make data publicly available.
Present availability requires mechanisms for access; future availability also requires preservation (e.g. long-term storage, format migration)\cite{waters_preserving_1996, beagrie_digital_2008, gray_online_2002}.
As in print publication, published data may not be free or legally unencumbered.
Data use agreements constrain many published datasets.
If access is limited, it should be contingent on clear and objective criteria; writing the creator for permission should not be part of the process.
For example, before granting access to restricted data, \href{http://www.icpsr.umich.edu/icpsrweb/content/deposit/confidentiality.html}{The Inter-university Consortium for Political and Social Research (ICPSR)} judges the applicant's proposed security measures, but not the merit of their research.
Datasets from social science or clinical studies that involve human participants are easily the most common source of access restrictions because of the need to protect privacy.

As a practical matter, publishing a dataset usually includes depositing it in a trustworthy repository.
While some aspects can be agreed on, what constitutes a ``trustworthy'' repository is somewhat subjective, and there are a handful of certification schemes to choose from.
In 2007, The Center for Research Libraries published the most extensive scheme: the Trusted Repository Audit Checklist (TRAC)\cite{dale_trustworthy_2007}.
Many repositories consult TRAC for self-assessment, but only four have completed the lengthy and rigorous process to be officially certified.
The process to obtain a \href{http://datasealofapproval.org/}{Data Seal of Approval (DSA)} is considerably more streamlined.
The DSA guidelines were also first released, by The Dutch \href{http://www.dans.knaw.nl/en}{Data Archiving and Networked Services (DANS)}, in 2007; 24 repositories have been stamped with the DSA since then.
Few of the hundreds of repositories in operation (e.g, the 973 now listed \href{http://databib.org/}{Databib} or the 609 at \href{http://www.re3data.org/}{re3data.org}) have pursued any kind of certification.
Given the low adoption of repository certification, a more typical way to decide trustworthiness is to judge by the organization responsible.
Repositories run by governments or large universities are likely to be considered trustworthy (although the effects of the 2013 US government shutdown on the PubMed biomedical article database\cite{check-hayden_nih_2013} might give one pause).
In contrast, independent repositories such as figshare may be riskier in the long term.

\section*{Citability}\label{citability}

Data citation is the element of publication that has come the farthest toward consensus.
This year, a coalition– including Future Of Research Communication and E-Scholarship (FORCE11)\cite{bourne_improving_2012}, the Committee on Data for Science and Technology (CODATA)\cite{codata-icsti_task_group_on_data_citation_standards_and_practices_out_2013}, the Digital Curation Centre (DCC), and others– released a \href{http://www.force11.org/datacitation}{Joint Declaration of Data Citation Principles}.
The first of the eight principles states, in part, that ``[d]ata citations should be accorded the same importance in the scholarly record as citations of other research objects, such as publications''.
Most of the time, this means that when a published dataset contributes to a paper, it should be cited formally in the reference list.
	
Data publishers enable formal citation by assigning unique permanent identifiers, most commonly the same ones used for journal articles: a Digital Object Identifiers (DOIs).
In addition to clarifying exactly what resource is being cited, a DOI can be resolved to locate the referenced dataset.
Note, however, that a DOI is neither sufficient nor necessary for citability- if a dataset moves and the DOI is not updated, the citation breaks and, conversely a well-maintained URL works as well as a DOI.


\subsection*{Simple Case}\label{simple-case}

The present consensus is that a dataset should be cited using, at a minimum, five elements largely familiar from article citations: creator(s), title, year, publisher and identifier.
This format agrees with CODATA's recommendation\cite{codata-icsti_task_group_on_data_citation_standards_and_practices_out_2013} and conveys all the information required to obtain a \href{http://www.datacite.org/}{DataCite} DOI\cite{datacite_datacite_2013} or be listed in the \href{http://thomsonreuters.com/data-citation-index/}{Thomson-Reuters Data Citation Index}.
However, this article-derived formulation fails to address some of the complications unique to datasets, described below.

\subsection*{Deep Citation}\label{deep-citation}

The first major complication that data citation faces is the need for \textbf{deep citation}.
When supporting an assertion in writing, it is usually sufficiently precise to cite the entirety of a journal article and leave it to the suspicious reader to find the relevant passage.
To reproduce an analysis performed on a subset of a larger dataset, the reader needs to know exactly what subset was used (e.g. a limited range of dates, only the adult subjects, wind speed but not direction).
Datasets vary so much in structure that there may not be a good general solution.
The most common suggestion is to cite the entire dataset in the reference list and describe the subset in the text of the paper \cite{altman_a_2007}.
In straightforward cases, the \href{http://wiki.esipfed.org/index.php/Interagency_Data_Stewardship/Citations/provider_guidelines#Subset_Used}{Federation of Earth Science Information Partners (ESIP)} and the \href{http://nsidc.org/about/use_copyright.html}{National Snow and Ice Data Center (NSIDC)} both suggest including a list of variables or range of dates in the formal citation.

\subsection*{Dynamic Datasets}\label{dynamic-data}

The second major complication arises when a dataset changes.
In the past, the printing process cemented a single version of an article as the final version of record.
Web-based publishing and preprint servers (e.g. \href{http://arxiv.org/}{arXiv.org}) complicate the situation for literature, but datasets are especially prone to be \textbf{dynamic}.
Consider, for example, a dataset collected by sensors in a weather station that expands every time a measurement is recorded.
Or, consider a dataset composed of predicted mRNAs drawn from an organism's genome that is periodically revised based on new experimental data and improved prediction methods. %might be good to cite a real dataset here... actual weather station data, or satellite data website...
Being able to update and correct datasets is extremely valuable.
However, to reproduce a data analysis, the reader needs not only the precise subset, but also the precise version.
If the dataset is dynamic, that means that previous versions must be preserved and citable.

Two kinds of dynamic datasets warrant consideration: \textbf{growing} datasets (e.g. the weather station readings) that add new data without ever changing or deleting existing data, and \textbf{revisable} datasets (e.g. the mRNAs) where data may by added, deleted, or changed.  %there's a citation for this somewhere (good! not sure which it is... you'd know better than me)
Growing datasets can be cited with an access date or a date range in the citation.
Revisable datasets are more difficult; the most common approach is to accumulate revisions and periodically publish a new version with a citable version number. %again - an example or citation of a group that does this?

Controversy around dynamic data and identifiers persists and publishers have different policies.
DataCite recommends, but does not require, that their dataset DOIs reference immutable datasets.
In contast, the DCC, for example, recommends periodically issuing growing datasets a new DOI that refers to the ``time-slice'' of new records and periodically freezing revisable datasets as a ``snapshot'' with a new DOI.

\subsection*{Just-in-time Identifiers}\label{just-in-time-identifiers}

Deep citation and dynamic data could potentially both be solved by turning the identifier-issuing process on its head.
Instead of the dataset publisher issuing identifiers for data at the level that researchers seem likely to cite, researchers could issue identifiers for precisely the part of the dataset that they want to cite.
The Research Data Alliance (RDA) \href{http://rd-alliance.org/working-groups/data-citation-wg.html}{Data Citation Working Group} recently put forth a sophisticated proposal for data in (or convertible to) databases.
Identifiers created under this scheme would wrap together identification of a database, a query to return the cited dataset, the version of the database queried for this analysis, and a number of other useful components.
This seems promising, but there are still many technical and policy issues that have to be resolved before this can be widely adopted.

\section*{Validation}\label{validation}

Data validation is the least resolved aspect of data publication, and fundamental questions are still unanswered:
What minimum guarantee of quality should a published dataset have?
How and by what criteria can datasets be evaluated against that guarantee?
Is literature peer-review an appropriate model?

Callaghan (2012)\cite{sarah_callaghan_making_2012} draws a useful distinction between \textbf{technical} and \textbf{scientific} review.
Technical review verifies that a dataset is complete, the description is complete, and that the two match up.
Domain expertise is generally not required, and many repositories provide at least some level of technical review.
Scientific review evaluates the methods of data collection, the overall plausibility of the data, and the likely reuse value.
Scientific review does require domain expertise, making this level of validation more difficult to organize, and few repositories provide it.
On approach is for review of publications with data papers to be split between the repository for technical review and the data journal for scientific review.

\subsection*{Data paper peer-review}\label{data-paper-peer-review}

Peer-review guarantees that journal articles entering the scholarly record reach some level of validity (although the reproducibility crisis calls into question exactly what that level is).
In many fields, peer-reviewed publications enjoy a much higher status than even the most reputable grey literature.
The effort to apply the prestige of ``publication'' to datasets cascades naturally into an effort to apply the prestige of ``peer-review.''
But as data validation seeks to model itself on literature peer-review, literature peer-review itself is in flux\cite{pulverer_transparent_2010, herron_is_2012, kriegeskorte_emerging_2012}.
Open peer-review at \emph{F1000Research} and post-publication commenting at \href{http://www.ncbi.nlm.nih.gov/pubmedcommons/}{PubMed Commons} are just two of many ongoing web-enabled experiments in article evaluation.

Journal article reviewers traditionally consider whether the methods used are appropriate for the questions asked and the data collected support the conclusions drawn.
In the absence of particular questions and conclusions, it is unclear what peer-review of data should certify.
A dataset may be suitable for some purposes, but not for others.\cite{parsons_data_2010}
Another issue is that while a reviewer can be expected to read an entire article, they cannot inspect every point in a large dataset.
Finally, researchers are already overwhelmed by peer-review of articles\cite{diederich_are_2013}, so increasing their workload may be unreasonable.
Despite these difficulties, venues for peer-reviewed data papers are opening rapidly.

Data paper journals wrap scientific peer-review of the paper and the dataset together into a single process. %meg: Needs transition, not sure how we got here
\emph{GigaScience}, an exception, assigns a separate data reviewer for technical review of the dataset.
Journals provide roughly similar guidelines to reviewers guidelines, with the exception that roughly half of current data journals consider novelty or potential impact, while the others only require that the dataset be scientifically sound.
In contrast, review processes differ widely.

As an example, consider \emph{Scientific Data} and \emph{Biodiversity Journal}.
Both journals divide reviewer guidelines into three sections along similar lines, which \emph{Biodiversity Journal} calls ``quality of the data'', ``quality of the description'', and ``consistency between manuscript and data.''
\emph{Scientific Data} follows a traditional peer-review process: an editor appoints reviewers who are encouraged to remain anonymous.
In contrast, review at \emph{Biodiversity Journal} follows a flexible and open process featuring entirely optional anonymity and multiple types of reviewer.
There, an editor appoints two or three ``nominated'' reviewers who must report back and several ``panel'' reviewers who read the paper and only comment at their own discretion.
Additionally, the authors may choose to open the paper to public comment during the review process.
Review processes for standalone datasets are less clearly defined.


\subsection*{Independent data validation}

Data journals all model their data validation more or less faithfully on literature peer-review, but independent data validation practices and proposals are considerably more varied.
On the conservative end of the spectrum, Lawrence (2011) proposes a set of criteria for independent data peer-review.\cite{lawrence_citation_2011}
The \href{https://pds.jpl.nasa.gov/}{Planetary Data System (PDS)}\cite peer-reviews datasets through the unusual process of holding an in-person meeting with representatives of the repository, the dataset creators, and the reviewers.

Open Context publishes archaeology datasets with extensive (although optional) review processes that incorporate peer-review in a way that goes beyond the simple accept/reject binary.\cite{kansa_we_2013}
Each dataset is rated from one to five circles based not on quality \emph{per se}, but how thoroughly it has been reviewed.
A three indicates that the dataset has passed a technical review, a four that it has passed editorial review, and a five that it has passed external peer-review.
In contrast, \href{http://www.tdar.org/}{The Digital Archaeological Record (tDAR)} makes data deposit easy by requiring minimal description (although it accommodates much more) and performing only technical.
(Although \emph{Internet Archaeology} and \href{http://openarchaeologydata.metajnl.com/}{\emph{Journal of Open Archaeological Data}} both peer-review tDAR datasets associated with data papers.)  %meg: I am lost
tDAR and Open Context differ in both mission and scale; tDAR is a large scale repository primarily concerned with collecting and preserving archaeology data for future use, while Open Context operates on a much smaller scale to publish data for easy reuse.
Thus, the appropriate validation for a dataset depends not only on the type of data and the discipline, but also on organizational goals and intended use.

Pre-publication validation can supplemented or replaced by post-publication feedback from successful or unsuccessful reusers.
Parsons et al. (2011) suggest that ``data use in its own right provides a form of review,'' and go on to point out that the context of reuse, demonstrates a that the data is not just vaguely ``good'', but fit for some particular purpose \cite{parsons_data_2011}.
The DANS repository solicits feedback from researchers who use its datasets: users are asked to rate the dataset on a one to five scale in each of six criteria (e.g., data quality, quality of the documentation, structure of the dataset) \cite{grootveld_data_2011,grootveld_peer-reviewed_2012}.


\section*{Beyond data publication}\label{beyond-data-publication}

In a 2013 paper\cite{parsons_is_2013}, Parsons and Fox argue that thinking about data through the the metaphor of print ``publication'' is limiting.
Diverse kinds of material are regarded as data by one research community or another and, while at least some aspects of publication apply well to at least some kinds of data, other approaches are possible. %meg: Your point about peer-review failing for literature might go better in this section
An alternative metaphor that seems to be gaining traction is ``data as software''\cite{schopf_treating_2012}.
In some cases, it may be better to think of releasing a dataset as one would a piece of software and to regard subsequent changes as analogous to updated versions.
The open source software community has already developed many potentially relevant tools for working collaboratively, managing multiple versions, and tracking attribution.
Ram (2013)\cite{ram_git_2013} catalogs a multitude of scientific uses for the software version control system \href{http://git-scm.com/}{Git}, including for managing data.
Open context uses Git and \href{http://www.mantisbt.org/}{Mantis Bug Tracker} to track and correct dataset errors.
Furthermore, projects such as \href{http://ipython.org/notebook}{IPython Notebook} integrate data, processing, and analysis into a single package.
However, scientific software struggles for recognition\cite{pradal_publishing_2013} just as data does, so using it to alter or affect the academic reward system for data is a tricky prospect.
Ultimately, while ``data as software'' is promising, data is not sofware. Nor is it literature.
The prestige and familiarity of terms like ``publication'' and ``peer-review'' are powerful, but we may have to stretch their definitions if we are determined to apply them to data.

%\nocite{*}
{\small\bibliographystyle{unsrt}
\bibliography{LandscapePaper}}



% See this guide for more information on BibTeX:
% http://libguides.mit.edu/content.php?pid=55482&sid=406343

% For more author guidance please see:
% http://f1000research.com/author-guidelines


% When all authors are happy with the paper, use the
% ‘Submit to F1000Research' button from the Share menu above
% to submit directly to the open life science journal F1000Research.

% Please note that this template results in a draft pre-submission PDF document.
% Articles will be professionally typeset when accepted for publication.

% We hope you find the F1000Research writeLaTeX template useful,
% please let us know if you have any feedback using the help menu above.


\end{document}