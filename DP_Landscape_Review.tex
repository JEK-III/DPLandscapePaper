% F1000Research template from writeLaTeX

%\documentclass[10pt,a4paper,twocolumn]{article}
\documentclass[10pt,twocolumn]{article}

\usepackage[superscript]{cite}
\usepackage{f1000_styles}
\usepackage{hyperref}

\hypersetup{colorlinks=true}


\begin{document}


\title{The Data Publication Landscape.}
\author[1]{John Ernest Kratz}
\author[1]{Carly Strasser}
\affil[1]{California Digital Library, University of California Office of the President, Oakland, CA 94612, USA}

\maketitle
\thispagestyle{fancy}

% Please list all authors that played a significant role in the research involved in the article. Please provide full affiliation information (including full institutional address, ZIP code and e-mail address) for all authors, and identify who is/are the corresponding author(s).

\begin{abstract}


The movement to incorporate datasets into the scholarly record as first class\cite{sarah_callaghan_making_2012} research products (validated, preserved, cited, and credited) has been slowly building momentum for some time, but the pace of developments picked up substantially in the last year.
Data publications are proliferating, but there are still significant debates over formats, processes, and terminology.
This article will present an overview of the initiatives underway and the current conversation, highlighting places where consensus seems to have been reached and issues still in contention.

Data publications follow a variety of models that differ in, among other things, what kind of documentation is published, where the data lives relative to the documentation, and how the data is validated.
Data can be published as supplemental material to a journal article, with a descriptive ``data paper'', or independently.
Further complicating the situation, the same terms are used by different initiatives to refer to related by distinct concepts.
The term `published' means that the data is public and citable, but it may or may not mean peer reviewed.
In turn, data `peer review' can refer to substantially different processes, although data paper referee guidelines are fairly uniform.
There is substantial agreement on the elements of a dataset citation (which closely resembles that of a journal article) but a variety of solutions for citing subsets of datasets or datasets that change over time.
Finally, some are already looking past data publication to other metaphors, such as `data as software', for solutions to unsolved problems.

\end{abstract}
\clearpage

\section*{Introduction: what does data publication mean?}\label{introduction}
%\subsection*{What does ``data publication'' mean?}\label{what-does-data-publication-mean}

The idea that researchers should share data to advance knowledge and promote the common good is not new, but in recent years the conversation has shifted from sharing data to ``publishing'' data.\cite{costello_motivating_2009,smith_data_2009,lawrence_data_2011,sarah_callaghan_making_2012}
The shift in language reflects the belief that datasets should be brought into the scholarly record and afforded the same ``first class'' status as traditional research products like journal articles.  
While this goal is widely shared,  ``data publication'' has become something of a platitude, with different people and organizations implying different things when using the phrase.

Within the scholarly communication community, two properties of a data publication are widely agreed on.\cite{smith_data_2009,sarah_callaghan_making_2012}
Firstly, published data is \textbf{available} now and for the indefinite future, without gatekeeping by the creator (although access may be limited by subscription or acceptance of a use agreement).
Secondly, it is formally \textbf{citable} like a journal article.
Less agreed on is a third property: how and to what extent published data must be shown to be \textbf{valid}.
Callaghan (2012)\cite{sarah_callaghan_making_2012} draws a distinction between data that has been shared, published (note the lower-case “p”), or Published (note the upper-case “P”): \textbf{shared} data is available, \textbf{published} data is available and citable, and \textbf{Published} data is available, citable, and validated. 
In practice, availability is usually satisfied by depositing the dataset in a repository, citability by assigning a persistent identifier (e.g. a Digital Object Identifier), and validity by peer-review.

\subsection*{Why publish data?}\label{why-publish-data}

The underlying goals of data publicaiton are to make research more \textbf{reproducibile} and to make \textbf{reuse} of data possible.
Science is suffering from a percieved ``reproduciblility crisis.''\cite{mobley_survey_2013,pashler_is_2012,zimmer_rise_2012,hiltzik_science_2013,begley_drug_2012}
The problem is exacerbated by lack of access to primary data, which enables fraud and honest errors to go undetected.
Data publication could help to address the reproducibility crisis by increasing the chance of detecting errors and reducing the opportunity for fraud.\cite{drew_lost_2013}
In a recent editorial on reproducibliity in science, the leaders of the US National Institutes of Health (NIH) cited ``provid[ing] greater transparency of the data that are the basis of published manuscripts'' as a way to improve reproducibility.\cite{collins_nih_2014}

%add reuse


\section*{Types of data publication}\label{types-of-data-publication}

At present, the still-solidifying phrase ``data publication'' covers a number types of research objects published via a variety of processes.
Depending on who's speaking,  data publication might be an excel spreadsheet on a website, a set of images deposited in an institutional archive, a stream of readings from a weather station available via the internet, or a peer-reviewed article describing a dataset hosted elsewhere.
Given the huge variety of types of data, it seems unlikely that any single structure will be ideal for every discipline and every dataset, but we can hope for a manageable number of blueprints. 
Lawrence (2011) lays out a taxonomy of five data publication models ``discriminated in the main by how the roles involved in publication are distributed between the various actors'' (e.g. the author, archive or journal).\cite{lawrence_data_2011}
For the purposes of this paper, we will more simply classify data publications into three categories based on the accompanying documentation; a dataset may \textbf{supplement} a traditional research paper, be the \textbf{subject} of a ``data paper'', or be \textbf{independent} of any paper.

%need a ref for variety of types of data

\subsection*{Data that supplements a paper}\label{paper-supplement-data}

The most familiar model of data publication to researchers is data published along with a traditional journal article. 
This can be in the form of either spplementary material (hosted by the journal publisher) or in a third-party repository.
Third-party repositories are generally consiered to be better suited to ensure long-term preservation and access to the data.
The Journal of Neuroscience stopped publishing supplemental material in 2010.\cite{maunsell_announcement_2010}
The most prominent third-party repository specifically for making supplemental data public is \href{http://www.datadryad.org/}{Dryad}, which accepts data underlying any peer-reviewed or otherwise ``reputable'' publication. 
Dryad makes data available and citable, but any assessment of scientific validity must be managed by the publisher of the article.
Other examples of third-party repositories include \href{http://figshare.com/}{figshare}, institutional repositories, and discipline-specific repositories, e.g. DNA sequences are deposited in \href{http://www.ncbi.nlm.nih.gov/genbank/}{GenBank}\cite{benson_genbank_2013} and protein structures in the \href{http://www.rcsb.org/}{Protein Data Bank}\cite{berman_protein_2000}

Journal publishers frequently require that the data underlying the figures and results in an article be furnished to interested parties on request.
A 2011 survey of author instructions at 50 high-impact publications found that 88\% included a statement regarding the availability of underlying data, and half of those made willingness to provide data a condition of publication.\cite{alsheikh-ali_public_2011}
However, the same study found that 59\% of papers ``did not fully adhere to the data availability instructions''.
Even more alarming, when Vines et al. attempted obtain data from 516 papers published between 1991 and 2000, they could only confirm the existence of the data in 23\% of the total cases.\cite{vines_availability_2014}
Now, journals are beginning to require that the data be published, not just made available on request.
In 2010, a coallition of Ecology and Evolutionary Biology journals began to require that the data underlying articles be archived at the time of publication, with a maximum embargo of 1 year.\cite{whitlock_data_2010, fairbairn_advent_2010}
{\emph{The Public Library of Science (PLOS)}} journals followed suit this year (without an embargo period)\cite{bloom_data_2014}, and \href{http://f1000research.com}{\emph{F1000Research}} has had a similar policy since its inception.


% include Vines and reorganize
 
\subsection*{Data as the subject of a paper}\label{paper-subject-data}

A \textbf{data paper} describes a dataset by thoroughly detailing the rationale and collection methods, but lacks any analysis or conclusions.\cite{newman_data_2009} 
Data papers are flourishing as a new article type in journals such as \emph{F1000Research}, \href{http://www.internetarchaeology.org/}{\emph{Internet Archaeology}}, and \href{http://www.gigasciencejournal.com/}{\emph{GigaScience}}\cite{gigascience}, and in new journals dedicated to the format like \href{http://onlinelibrary.wiley.com/journal/10.1002/%28ISSN%292049-6060}{\emph{Geoscience Data Journal}}\cite{geoscience_data_journal}, a trio of ``metajournals'' from Ubiquity Press, and Nature Publishing Group's forthcoming \href{http://www.nature.com/scientificdata/}{\emph{Scientific Data}}.

The length and structure of data papers varies significantly between journals, but the tendency is toward relatively short and structured papers.
All of them feature an abstract, collection methods, and description of the dataset; a few (e.g. \emph{Internet Archaeology}, \href{http://openhealthdata.metajnl.com/about/submissions#authorGuidelines}{\emph{Open Health Data}}) encourage authors to suggest potential uses for the data.
In addition to this general framework, some journals incorporate sections specific to the field; for instance, \emph{Internet Archaeology} and the \href{http://openarchaeologydata.metajnl.com/}{\emph{Journal of Open Archaeology Data}} both include a separate section to describe the  temporal or geographic scope of the dataset.
Data papers are best delineated from traditional articles not by the presence of any particular information, but by the absence of analysis and conclusions.
A sharp distinction is needed because many publishers (e.g. ones on a \href{https://f1000research.com/data-policies}{list} complied by \emph{F1000Research}) do not consider a data paper as a prior publication, should the authors seek to publish a subsequent analysis paper.

With few exceptions, data journals require datasets to be published by a trusted third-party repository. 
\emph{GigaScience} has a tightly associated repository, \emph{GigaDB}, to host datasets, and \emph{The International Journal of Robotics Research}\cite{international_journal_of_robotics_research}, an early comer to data papers\cite{newman_data_2009}, allows authors to host datasets on their own websites, but more typcially, \emph{Scientific Data} and \emph{Geoscience Data Journal} list approved disciplinary and general-purpose third-party repositories in their instructions for authors.

\subsection*{Data independent of any paper}\label{paper-independent-data}

To be useful or reproducible, a dataset must have accompanying descriptive information (i.e. metadata), but this needn't take the form of a journal article. 
Datasets can be published by a repository, or with rich structured or freeform description colocated with the dataset instead of an associated journal article.
Repositories are able to provide access and citability, but the degree of validation varies widely. 
Few are equipped to provide peer-review. Figshare, for instance, publishes datasets without any validation (although a Figshare dataset associated with a data paper will have been reviewed along with the paper). 
On the other hand, \href{http://opencontext.org/}{Open Context} publishes very high quality archeology datasets with optional peer review.

% extra examples, fleshing out definitely needed


% \section*{How does publication work?}\label{how-does-publication-work}

\section*{Availability}\label{availability}

To publish is to make public; at its most fundamental, to publish data is to make data public. 
For published data to be available in the future, both preservation and access mechanisms are required.\cite{beagrie_digital_2008, gray_online_2002}
As in print publication, there is no requirement that published data be free or legally unencumbered.
However, if access is limited, there should be clear and objective criteria for access, which must then be granted to anyone who satisfies them.
The most common need for access restrictions restrictions is when the dataset contains data from human subjects.
Writing the creator for permission must notr be part of the process.

As a practical matter, publishing a dataset usually means depositing it in a trustworthy repository. 
What constitutes a trustworthy repository is largely subjective, but some measures can be agreed on. 
Multiple certification schemes exist.
The most extensive is the Trusted Repository Audit Checklist (TRAC)\cite{dale_trustworthy_2007} from the Center for Research Libraries, but TRAC certification is so onerous that only four repositories have gone thorough the process. 
The \href{http://datasealofapproval.org/}{Data Seal of Approval} has been awarded to 24 repositories following a considerably more streamlined process.
A more typical way to decide trustworthiness is to judge by the organization running it. 
Repositories run by governments or large universities are likely to be considered trustworthy (although the effects of the 2013 US government shutdown on PubMed might give one pause).
In contrast, independent repositories such as figshare and Dryad may be riskier in the long term. %a wise thing to say?  actual failure might be better, but don't know of any

\section*{Citability}\label{citability}

Citation is perhaps the element of data publication that has come the farthest toward establishing consensus.  
This year, a coalition of groups involved in scholarly communication– including Future Of Research Communication and E-Scholarship (FORCE11)\cite{bourne_improving_2012}, the Committee on Data for Science and Technology (CODATA)\cite{codata-icsti_task_group_on_data_citation_standards_and_practices_out_2013}, the Digital Curation Center (DCC), and others– released a \href{http://www.force11.org/datacitation}{Joint Declaration of Data Citation Principles}.
The first of the 8 principles states that 
``Data should be considered legitimate, citable products of research. Data citations should be accorded the same importance in the scholarly record as citations of other research objects, such as publications''
Most of the time, this means that if researcher uses a published data set in a paper, they should cite the dataset formally in the reference list. 
	
Data publications generally enable citation by assigning a unique permanent identifier to the dataset.
Most commonly, this is the same identifier used for journal articles, a Digital Object Identifier (DOI).
For as long as the DOI is maintained, anyone interested can use it to locate the dataset. 
Note, however, that a DOI is neither sufficient nor necessary for citability-- if the DOI is not maintained, the citation breaks and, conversely a well maintained URL works as well as a DOI. 

\subsection*{Simple Case}\label{simple-case}

In the simplest case, there is substantial agreement that a published dataset should be cited using five elements largely familiar from journal citations: creator(s), title, year, publisher and identifier. 
This format is consistent with the example accompanying the Joint Declaration of Data Ciation Principles, the recommendation made by CODATA\cite{codata-icsti_task_group_on_data_citation_standards_and_practices_out_2013} and with the information requried to obtain a DataCite DOI\cite{datacite_datacite_2013} or be listed in the \href{http://thomsonreuters.com/data-citation-index/}{Thomson-Reuters Data Citation Index}.
However, this article-descended formulation is not adequate to address some of the complications unique to datasets.

\subsection*{Deep Citation}\label{deep-citation}

The first major complication that datasets face is the need for \textbf{deep citation}. 
When supporting an assertion in writing, it is considered sufficiently precise to cite the entirety of the referenced journal article and leave it to the suspicious reader to identify the basis of your assertion. 
If only part of a dataset is used in a quantitative analysis, one may need to specify exactly the subset in question. 
Because datasets are so variable in structure, a general solution may not exist. 
The most common suggestion is to cite the entire dataset and describe the subset in the text of the paper.\cite{altman_a_2007,}
In some cases, it may be practical to include a date or record number range or a list of variables in the formal citation, e.g. as suggested by the  \href{http://wiki.esipfed.org/index.php/Interagency_Data_Stewardship/Citations/provider_guidelines#Subset_Used}{Federation of Earth Science Information Parterns (ESIP)} or the \href{http://nsidc.org/about/use_copyright.html}{National Snow and Ice Data Center (NSIDC)}. 

\subsection*{Dynamic Data}\label{dynamic-data}

A second complication is that datasets are prone to existing in multiple versions or changing over time. 
In the past, the printed article was a single version of record.
Web based publishing and preprint servers such as \href{http://arxiv.org/}{arXiv.org} have already complicated the matter. 
Data publishers are likely to allow or even encourage updating and correction of datasets. 
For the results of data analysis to be reproducible, the reader must be able to obtain precisely the version of the data that the researcher used. 
In the case of \textbf{dynamic data}, that means that previous versions have to be preserved and citable.

As a practical matter, there are two kinds of dynamic data that warrant consideration: \textbf{expanding} datasets, to which new data may be added but old data will never be changed or deleted, and \textbf{revisable} datasets in which data may be added, deleted, or changed over time. 
Common solutions to expanding data are to include an access date, or a date or record number range in the citation. 
Revisable datasets are more difficult, but the most common approach is to periodically publish multiple changes as a new version with a version number that can be included in citations.

Controversy persists about dynamic data and identifiers and different publishers have different policies. 
DataCite recommends but does not require that the DOIs that they issue point to immutable objects. 
The DCC, for example, recommends that growing datasets be issued a new DOI periodically that refers to the ``time-slice'' of records added since the last DOI was issued; revisable datasets are to be periodically frozen as a ``snapshot'' and issued a new DOI.

\subsection*{Just-in-time Identifiers}\label{just-in-time-identifiers}

One potential solution to both deep citation and dynamic data is to turn the identifier-issuing process on its head. 
Instead of a dataset publisher minting the identifier, the researcher who wants to cite a dataset could mint an identifier that refers to precisely the part of the dataset that they wish to cite. 
The Research Data Alliance (RDA) \href{http://rd-alliance.org/working-groups/data-citation-wg.html}{Data Citation Working Group} has put forth a sophisticated proposal suitable for databases in which an identifier would wrap together a number of components including specifying a version of the database and a query over the database that produces the cited dataset. 
This seems promising, but there are still many technical and policy issues that have to be resolved before this can be widely adopted.

\section*{Validation}\label{trustworthiness}

Callaghan (2012)\cite{sarah_callaghan_making_2012} draws a useful distinction between \textbf{technical} and \textbf{scientific} review. 
Technical review verifies that the dataset is complete, the description is complete, and that the two match up.  
Technical review generally doesn't require domain expertise, and many repositories provide at least some level of review. 
Scientific review evaluates the methods of data collection, the overall plausibility of the data, and the likely reuse value. 
Scientifc review requires domain expertise and is more difficult to organize, so few repositories provide it.
In the case of a data paper, it's common for the repository to do the technical review and the data journal to do the scientific review.

\subsection*{Prepublication peer-review}

For journal articles, peer-review is the gatekeeper to the scholarly record, meant to ensure some level of validity.
In many fields formally peer-reviewed literature enjoys a much higher status than even the most reputable grey literature.
The effort to apply the prestige of ``publication'' to datasets cascades into an effort to apply the prestige of ``peer-review'' to data.
Like data publication, data peer-review is being defined now.

\subsubsection*{Paper subject data review}
Publishers of data papers wrap peer review of the paper and of the dataset together. 
An exception is \emph{GigaScience}, which assigns a separate data reviewer for technical review of the dataset. 
Reviewer guidelines are roughly similar across journals, although roughly half of the journals we looked at consider novelty or potential impact, while the others only require that the dataset be scientifically sound.
While review guidelines are similar, review processes are not. 
Data paper peer review processes range from traditional to experimental (open post-publication review in F1000 Research).

As an example, we can compare \emph{Scientific Data} and \emph{Biodiveristy Journal}.
The two journals each divide reviewer guidelines into three similar sections– quality of the data, quality of the description, and consistency between the description and the data– and provide similar guidance.
However, their peer review processes are quite different.
\emph{Scientific Data} implements a traditional peer review process: the editor appoints 1 or more reviewers, who are encouraged to remain anonymous.
\emph{Biodiversity Journal} has a more flexible and open process.
There, anonymity is up to each reviewer, and there are multiple classes of reviewer.
The editor appoints two or three ``nominated'' reviewers who are required to supply feedback and several ``panel'' reviewers who read the paper and only supply feedback if they feel like they have something to say.
Additionally, the authors may opt to open the paper to public comment during the review process.


\subsubsection*{Independent data review}
Review processes for standalone datasets are less clearly defined. 

%suggestion for data peer review\cite{lawrence_citation_2011}

NASA Planetary Data System (PDS) conducts peer review in an in-person meeting with representatives of the repository, the dataset creators, and the reviewers. 

\subsection*{Postpublication review}


Open Context goes beyond the simple accept/reject binary of traditional peer review.\cite{kansa_we_2013} 
Instead, each dataset has a rating from 1-5 that indicates how thoroughly it has been reviewed. 
Essentially, a 3 indicates that the dataset has passed technical review, a 4 means that it has passed editorial review, and a 5 means that it has passed external peer review. 

The Dutch \href{http://www.dans.knaw.nl/en}{Data Archiving and Networked Services (DANS)} solicts structured, multifaceted feedback from users of their datasets: users are asked to assign a rating on a five star scale for each of six criteria (e.g., data quality, quality of the documentation, structure of the dataset).\cite{grootveld_data_2011,grootveld_peer-reviewed_2012} 


% \subsection*{Post-publication review}\label{post-publication-review}

\section*{Beyond data publication}\label{beyond-data-publication}

In a 2013 paper\cite{parsons_is_2013}, Parsons and Fox argue that thinking about data through the the metaphor of print ``publication'' is potentially very limiting.
Diverse kinds of material are regarded as data by one research community or another, and while at least some aspects of publication apply well to at least some kinds of data, there are many other possible approaches.
One alternative metaphor that seems to be gaining traction is data as software.\cite{schopf_treating_2012}

In some cases, it may be better to think of releasing a dataset as one would a piece of software, and to regard subsequent changes are analogous to updated versions.
The open source software community has already developmed many tools for working collaboratively, managing mutliple versions, and tracking attribution.
Ram (2013)\cite{ram_git_2013} catalogs a multitude of scientific uses for the software version control system \href{http://git-scm.com/}{Git}, including for managing data.
Open context came as a practical matter to use Git and \href{http://www.mantisbt.org/}{Mantis Bug Tracker} to track and correct dataset errors.
Furthermore, projects such as \href{http://ipython.org/notebook}{IPython Notebook} integrate data, processing, and analysis into a single package.
However, scientific software is struggling for recognition\cite{pradal_publishing_2013} just as data is, so revising the academic reward system continues to be a challenge.

Ultimately, while ``data as software'' is promising, data is neither literature nor software, and 
The prestige and familiarity of ``publication'' and ``peer-review'' are extremely useful, but it may be necessary to stretch the definitions of each to encompass processes that work for data.


%\nocite{*}
{\small\bibliographystyle{unsrt}
\bibliography{LandscapePaper}



% See this guide for more information on BibTeX:
% http://libguides.mit.edu/content.php?pid=55482&sid=406343

% For more author guidance please see:
% http://f1000research.com/author-guidelines


% When all authors are happy with the paper, use the 
% ‘Submit to F1000Research' button from the Share menu above
% to submit directly to the open life science journal F1000Research.

% Please note that this template results in a draft pre-submission PDF document.
% Articles will be professionally typeset when accepted for publication.

% We hope you find the F1000Research writeLaTeX template useful,
% please let us know if you have any feedback using the help menu above.


\end{document}